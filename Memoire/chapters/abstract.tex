%====================================================================
% Résumé - SecureIoT-VIF
%====================================================================

\chapter*{Résumé}
\addcontentsline{toc}{chapter}{Résumé}

\section*{Contexte}

L'Internet des Objets (IoT) connaît une croissance exponentielle avec 75 milliards d'appareils connectés attendus d'ici 2025, générant un marché mondial de plus de 6 000 milliards de dollars. Cette expansion s'accompagne d'une recrudescence alarmante des cyberattaques ciblant les firmwares IoT, avec 89\% des dispositifs présentant des vulnérabilités critiques. Les attaques par compromission de firmware permettent aux cybercriminels d'obtenir un contrôle persistant et de bas niveau, rendant les détections conventionnelles inefficaces.

\section*{Problématique}

Les solutions actuelles de sécurité des firmwares IoT présentent plusieurs limitations majeures : overhead computationnel excessif incompatible avec les dispositifs contraints, absence de vérification d'intégrité continue, complexité d'intégration dans les systèmes existants, et coût élevé de déploiement. Il existe un besoin critique de développer des solutions de sécurité robustes, performantes et adaptées aux contraintes spécifiques des environnements IoT industriels.

\textbf{Problem Statement:} Current IoT firmware security solutions fail to provide adequate protection against sophisticated attacks while maintaining compatibility with resource-constrained devices, lacking practical frameworks that balance security, performance, and deployability in industrial IoT environments.

\section*{Objectif}

Cette recherche vise à concevoir, développer et valider SecureIoT-VIF, un framework professionnel de vérification d'intégrité pour la sécurité des firmwares IoT. L'approche propose une architecture modulaire et évolutive implémentée sur plateforme ESP32, intégrant des mécanismes de cryptographie logicielle avancée, de vérification d'intégrité au démarrage, et de détection d'anomalies adaptative, avec une roadmap claire pour l'extension vers des fonctionnalités avancées (HSM matériel, vérification temps réel, machine learning).

\section*{Contributions}

\textbf{Contribution théorique :} Développement d'un modèle de sécurité hybride combinant vérification d'intégrité cryptographique et détection d'anomalies comportementales, optimisé pour les environnements IoT contraints tout en maintenant un niveau de sécurité adapté aux déploiements industriels.

\textbf{Contribution méthodologique :} 
\begin{itemize}
    \item Méthodologie d'implémentation de mécanismes de sécurité firmware sur ESP32 avec cryptographie mbedTLS
    \item Framework d'évaluation de la sécurité et des performances pour systèmes IoT contraints
    \item Architecture modulaire permettant l'intégration progressive de fonctionnalités avancées
\end{itemize}

\textbf{Contribution technique :} Implémentation complète d'un framework professionnel :
\begin{itemize}
    \item Vérification d'intégrité firmware avec signatures cryptographiques SHA-256/ECDSA
    \item Système de détection d'anomalies configurable avec seuils adaptatifs
    \item Architecture modulaire extensible à divers types de capteurs et actionneurs IoT
    \item Solution validée sur capteurs DHT22 (température/humidité) démontrant l'efficacité de l'approche
\end{itemize}

\textbf{Contribution empirique :} Évaluation expérimentale rigoureuse sur scénarios industriels représentatifs, avec analyse comparative des performances, démonstration de la faisabilité pratique et validation de l'efficacité de détection des attaques.

\section*{Résultats}

L'évaluation révèle des performances adaptées aux déploiements industriels : taux de détection élevé sur les attaques par compromission de firmware, overhead computationnel minimal (< 8\%), impact énergétique contrôlé, et architecture évolutive permettant l'intégration future de mécanismes avancés. Les tests avec capteurs DHT22 confirment l'efficacité de l'approche, démontrant la capacité du framework à s'intégrer avec divers types de capteurs IoT.

\section*{Impact}

Cette recherche établit une nouvelle approche pour la sécurité des firmwares IoT en proposant un framework professionnel qui équilibre efficacement sécurité, performance et praticité. L'architecture modulaire et la roadmap d'évolution vers des fonctionnalités avancées (HSM matériel, vérification temps réel, machine learning adaptatif) positionnent SecureIoT-VIF comme une solution pérenne pour les déploiements IoT industriels.

\textbf{Keywords:} IoT, Firmware Security, Professional Framework, ESP32, Hardware Security, Integrity Verification, Anomaly Detection, Industrial IoT, Embedded Systems, Cryptography

\vfill

\begin{center}
\textbf{Mots-clés :} IoT, Sécurité Firmware, Framework Professionnel, ESP32, Sécurité Matérielle, Vérification Intégrité, Détection Anomalies, IoT Industriel, Systèmes Embarqués, Cryptographie
\end{center}

%====================================================================
% Abstract English Version
%====================================================================

\chapter*{Abstract}
\addcontentsline{toc}{chapter}{Abstract}

\section*{Context}

The Internet of Things (IoT) is experiencing exponential growth with 75 billion connected devices expected by 2025, generating a global market exceeding \$6 trillion. This expansion is accompanied by an alarming surge in cyberattacks targeting IoT firmwares, with 89\% of devices presenting critical vulnerabilities. Firmware compromise attacks allow cybercriminals to gain persistent, low-level control, rendering conventional detection methods ineffective.

\section*{Problem Statement}

Current IoT firmware security solutions present several major limitations: excessive computational overhead incompatible with constrained devices, absence of continuous integrity verification, integration complexity into existing systems, and high deployment costs. There is a critical need to develop robust, performant security solutions adapted to the specific constraints of industrial IoT environments.

\section*{Objective}

This research aims to design, develop and validate SecureIoT-VIF, a professional integrity verification framework for IoT firmware security. The approach proposes a modular and scalable architecture implemented on ESP32 platform, integrating advanced software cryptography mechanisms, boot-time integrity verification, and adaptive anomaly detection, with a clear roadmap for extension toward advanced features (hardware HSM, real-time verification, machine learning).

\section*{Contributions}

\textbf{Theoretical contribution:} Development of a hybrid security model combining cryptographic integrity verification and behavioral anomaly detection, optimized for constrained IoT environments while maintaining security levels suitable for industrial deployments.

\textbf{Methodological contribution:}
\begin{itemize}
    \item Implementation methodology for firmware security mechanisms on ESP32 with mbedTLS cryptography
    \item Security and performance evaluation framework for constrained IoT systems
    \item Modular architecture enabling progressive integration of advanced functionalities
\end{itemize}

\textbf{Technical contribution:} Complete implementation of a professional framework:
\begin{itemize}
    \item Firmware integrity verification with SHA-256/ECDSA cryptographic signatures
    \item Configurable anomaly detection system with adaptive thresholds
    \item Modular architecture extensible to various IoT sensor and actuator types
    \item Solution validated with DHT22 sensors (temperature/humidity) demonstrating approach effectiveness
\end{itemize}

\textbf{Empirical contribution:} Rigorous experimental evaluation on representative industrial scenarios, with comparative performance analysis, practical feasibility demonstration and attack detection effectiveness validation.

\section*{Results}

The evaluation reveals performance suitable for industrial deployments: high detection rate for firmware compromise attacks, minimal computational overhead (< 8\%), controlled energy impact, and scalable architecture enabling future integration of advanced mechanisms. Tests with DHT22 sensors confirm the approach effectiveness, demonstrating the framework's capability to integrate with various IoT sensor types.

\section*{Impact}

This research establishes a new approach for IoT firmware security by proposing a professional framework that effectively balances security, performance and practicality. The modular architecture and evolution roadmap toward advanced features (hardware HSM, real-time verification, adaptive machine learning) position SecureIoT-VIF as a sustainable solution for industrial IoT deployments.

\textbf{Keywords:} IoT, Firmware Security, Professional Framework, ESP32, Hardware Security, Integrity Verification, Anomaly Detection, Industrial IoT, Embedded Systems, Cryptography
