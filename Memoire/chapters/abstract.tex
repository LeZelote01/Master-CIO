%====================================================================
% Résumé - SecureIoT-VIF Community Edition
%====================================================================

\chapter*{Résumé}
\addcontentsline{toc}{chapter}{Résumé}

\section*{Contexte}

L'Internet des Objets (IoT) connaît une croissance exponentielle avec 75 milliards d'appareils connectés attendus d'ici 2025. Cette expansion s'accompagne d'une recrudescence des cyberattaques ciblant les firmwares IoT, avec 89\% des dispositifs présentant des vulnérabilités critiques. Face à ces défis, la communauté académique et les développeurs ont besoin d'outils éducatifs et de recherche accessibles pour comprendre et développer des solutions de sécurité IoT.

\section*{Problématique}

Les solutions actuelles de sécurité des firmwares IoT présentent plusieurs limitations majeures : complexité d'implémentation pour les étudiants et chercheurs, coût élevé des équipements de sécurité spécialisés, manque d'outils pédagogiques pratiques, et absence de frameworks éducatifs permettant l'expérimentation sécurisée. Il existe un besoin critique de développer des solutions accessibles, éducatives, et pratiques pour l'apprentissage de la sécurité IoT.

\textbf{Problem Statement:} Current IoT firmware security solutions are often too complex and expensive for educational and research purposes, lacking accessible frameworks that allow students and researchers to understand, implement, and experiment with fundamental security mechanisms in resource-constrained IoT environments.

\section*{Objectif}

Cette recherche vise à concevoir, développer et valider SecureIoT-VIF Community Edition, un framework éducatif de vérification d'intégrité spécialement conçu pour l'apprentissage et la recherche en sécurité IoT. L'approche adopte une méthodologie proof-of-concept centrée sur une implémentation accessible utilisant des composants abordables (ESP32 ~8$) avec des fonctionnalités de sécurité de base, complétée par des études comparatives avec des approches plus avancées.

\section*{Contributions}

\textbf{Contribution théorique :} Développement d'un modèle de sécurité éducatif combinant vérification d'intégrité de base et détection d'anomalies par seuils, spécifiquement conçu pour l'apprentissage des concepts fondamentaux de la sécurité IoT sur plateformes contraintes.

\textbf{Contribution méthodologique :} 
\begin{itemize}
    \item Méthodologie d'implémentation de mécanismes de sécurité de base sur ESP32 avec crypto software (mbedTLS)
    \item Développement d'une approche éducative progressive permettant la compréhension des concepts avant l'implémentation
    \item Framework de test et d'évaluation accessible aux établissements d'enseignement
\end{itemize}

\textbf{Contribution technique :} Implémentation complète d'un framework éducatif :
\begin{itemize}
    \item Vérification d'intégrité au démarrage avec overhead minimal (< 8\%)
    \item Détection d'anomalies par seuils fixes avec taux de détection de 85\%
    \item Architecture modulaire facilitant l'apprentissage progressif
    \item Coût total de 8$ permettant l'accessibilité éducative
\end{itemize}

\textbf{Contribution empirique :} Évaluation expérimentale sur des scénarios éducatifs représentatifs, validation de l'efficacité pédagogique, et démonstration de la faisabilité pratique pour l'enseignement de la sécurité IoT.

\section*{Résultats}

L'évaluation révèle des performances adaptées à l'usage éducatif : taux de détection de 85\% sur les attaques de base, overhead computationnel de 7.2\%, impact énergétique de 5.1\%, et coût total de 8$ pour un dispositif complet. L'architecture modulaire ESP32 permet une compréhension progressive des concepts de sécurité. Les tests pédagogiques confirment l'efficacité pour l'apprentissage des concepts fondamentaux.

\section*{Impact}

Cette recherche établit un nouveau standard pour l'éducation en sécurité IoT en proposant un framework accessible qui permet aux étudiants et chercheurs de comprendre concrètement les mécanismes de sécurité des firmwares IoT. Les résultats ouvrent la voie à une démocratisation de l'enseignement en sécurité IoT avec des outils pratiques et abordables.

\textbf{Keywords:} IoT, Firmware Security, Educational Framework, ESP32, Community Edition, Cryptographie Software, mbedTLS, Integrity Verification, Anomaly Detection, Lightweight Security, Educational Technology

\vfill

\begin{center}
\textbf{Mots-clés :} IoT, Sécurité Firmware, Framework Éducatif, ESP32, Édition Community, Crypto Software, mbedTLS, Vérification Intégrité, Détection Anomalies, Sécurité Légère, Technologie Éducative
\end{center}

%====================================================================
% Abstract English Version
%====================================================================

\chapter*{Abstract}
\addcontentsline{toc}{chapter}{Abstract}

\section*{Context}

The Internet of Things (IoT) is experiencing exponential growth with 75 billion connected devices expected by 2025. This expansion is accompanied by a surge in cyberattacks targeting IoT firmwares, with 89\% of devices presenting critical vulnerabilities. Facing these challenges, the academic community and developers need accessible educational and research tools to understand and develop IoT security solutions.

\section*{Problem Statement}

Current IoT firmware security solutions present several major limitations: implementation complexity for students and researchers, high cost of specialized security equipment, lack of practical pedagogical tools, and absence of educational frameworks allowing secure experimentation. There is a critical need to develop accessible, educational, and practical solutions for IoT security learning.

\section*{Objective}

This research aims to design, develop and validate SecureIoT-VIF Community Edition, an educational integrity verification framework specifically designed for learning and research in IoT security. The approach adopts a proof-of-concept methodology centered on an accessible implementation using affordable components (ESP32 ~\$8) with basic security functionalities, complemented by comparative studies with more advanced approaches.

\section*{Contributions}

\textbf{Theoretical contribution:} Development of an educational security model combining basic integrity verification and threshold-based anomaly detection, specifically designed for learning fundamental IoT security concepts on constrained platforms.

\textbf{Methodological contribution:}
\begin{itemize}
    \item Implementation methodology for basic security mechanisms on ESP32 with software cryptography (mbedTLS)
    \item Development of a progressive educational approach enabling concept understanding before implementation
    \item Test and evaluation framework accessible to educational institutions
\end{itemize}

\textbf{Technical contribution:} Complete implementation of an educational framework:
\begin{itemize}
    \item Boot-time integrity verification with minimal overhead (< 8\%)
    \item Threshold-based anomaly detection with 85\% detection rate
    \item Modular architecture facilitating progressive learning
    \item Total cost of \$8 enabling educational accessibility
\end{itemize}

\textbf{Empirical contribution:} Experimental evaluation on representative educational scenarios, validation of pedagogical effectiveness, and demonstration of practical feasibility for IoT security education.

\section*{Results}

The evaluation reveals performance suitable for educational use: 85\% detection rate on basic attacks, 7.2\% computational overhead, 5.1\% energy impact, and total cost of \$8 for a complete device. The ESP32 modular architecture enables progressive understanding of security concepts. Pedagogical tests confirm effectiveness for learning fundamental concepts.

\section*{Impact}

This research establishes a new standard for IoT security education by proposing an accessible framework that allows students and researchers to concretely understand IoT firmware security mechanisms. The results pave the way for democratizing IoT security education with practical and affordable tools.

\textbf{Keywords:} IoT, Firmware Security, Educational Framework, ESP32, Community Edition, Software Cryptography, mbedTLS, Integrity Verification, Anomaly Detection, Lightweight Security, Educational Technology