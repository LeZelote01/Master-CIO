%====================================================================
% Résumé - SecureIoT-VIF
%====================================================================

\chapter*{Résumé}
\addcontentsline{toc}{chapter}{Résumé}

\section*{Contexte}

L'Internet des Objets (IoT) connaît une croissance exponentielle avec 75 milliards d'appareils connectés attendus d'ici 2025, générant un marché mondial de plus de 6~000 milliards de dollars. Cette expansion s'accompagne d'une recrudescence alarmante des cyberattaques ciblant les firmwares IoT, avec 89~\% des dispositifs présentant des vulnérabilités critiques. Les attaques par compromission de firmware permettent aux cybercriminels d'obtenir un contrôle persistant et de bas niveau, rendant les détections conventionnelles inefficaces.

\section*{Problématique}

Les solutions actuelles de sécurité des firmwares IoT présentent plusieurs limitations majeures : overhead computationnel excessif incompatible avec les dispositifs contraints, absence de vérification d'intégrité continue, complexité d'intégration dans les systèmes existants, et coût élevé de déploiement. Il existe un besoin critique de développer des solutions de sécurité robustes, performantes et adaptées aux contraintes spécifiques des environnements IoT industriels.

\textbf{Problem Statement:} Current IoT firmware security solutions fail to provide adequate protection against sophisticated attacks while maintaining compatibility with resource-constrained devices, lacking practical frameworks that balance security, performance, and deployability in industrial IoT environments.

\section*{Objectif}

Cette recherche vise à concevoir, développer et valider SecureIoT-VIF, un framework modulaire de vérification d'intégrité pour la sécurité des firmwares IoT. L'approche propose une architecture évolutive à deux configurations implémentée sur plateforme ESP32 : la \textbf{Configuration Standard}, implémentée et validée expérimentalement, intégrant des mécanismes de cryptographie logicielle, de vérification d'intégrité au démarrage et de détection d'anomalies configurable ; et la \textbf{Configuration Expert}, conçue en détail pour intégrer des mécanismes avancés (HSM matériel, vérification temps réel, machine learning adaptatif).

\section*{Contributions}

\textbf{Contribution théorique :} Développement d'un modèle de sécurité hybride combinant vérification d'intégrité cryptographique et détection d'anomalies comportementales, avec une architecture modulaire permettant une évolution progressive de la Configuration Standard vers la Configuration Expert.

\textbf{Contribution méthodologique :} 
\begin{itemize}
    \item Méthodologie d'implémentation de mécanismes de sécurité firmware sur ESP32 avec cryptographie mbedTLS
    \item Framework d'évaluation de la sécurité et des performances pour systèmes IoT contraints
    \item Architecture modulaire à deux configurations permettant l'intégration progressive de fonctionnalités avancées
\end{itemize}

\textbf{Contribution technique :} Implémentation complète de la Configuration Standard et conception détaillée de la Configuration Expert :
\begin{itemize}
    \item \textbf{Configuration Standard (implémentée)} : Vérification d'intégrité firmware avec signatures cryptographiques SHA-256/ECDSA, détection d'anomalies configurable avec seuils adaptatifs, architecture modulaire extensible, validation avec capteurs DHT22
    \item \textbf{Configuration Expert (conçue)} : Spécifications détaillées pour HSM matériel, vérification temps réel continue, machine learning adaptatif, attestation distante
\end{itemize}

\textbf{Contribution empirique :} Évaluation expérimentale rigoureuse de la Configuration Standard sur scénarios représentatifs, avec analyse comparative des performances, démonstration de la faisabilité pratique et validation de l'efficacité de détection des attaques.

\section*{Résultats}

L'évaluation de la Configuration Standard révèle des performances prometteuses : taux de détection élevé sur les attaques par compromission de firmware, overhead computationnel minimal (< 8\%), impact énergétique contrôlé, et architecture modulaire validée. Les tests avec capteurs DHT22 confirment l'efficacité de l'approche implémentée. La Configuration Expert, conçue en détail, établit une roadmap scientifiquement fondée pour l'intégration future de mécanismes avancés.

\section*{Impact}

Cette recherche établit une nouvelle approche pour la sécurité des firmwares IoT en proposant une architecture modulaire à deux configurations qui équilibre efficacement sécurité, performance et praticité. L'implémentation de la Configuration Standard démontre la faisabilité de l'approche, tandis que la conception détaillée de la Configuration Expert positionne SecureIoT-VIF comme une solution évolutive pour les déploiements IoT.

\textbf{Keywords:} IoT, Firmware Security, Modular Framework, ESP32, Integrity Verification, Anomaly Detection, Embedded Systems, Cryptography, Two-Tier Architecture, Standard Configuration, Expert Configuration

\vfill

\begin{center}
\textbf{Mots-clés :} IoT, Sécurité Firmware, Framework Modulaire, ESP32, Vérification Intégrité, Détection Anomalies, Systèmes Embarqués, Cryptographie, Architecture à Deux Configurations, Configuration Standard, Configuration Expert
\end{center}

%====================================================================
% Abstract English Version
%====================================================================

\chapter*{Abstract}
\addcontentsline{toc}{chapter}{Abstract}

\section*{Context}

The Internet of Things (IoT) is experiencing exponential growth with 75 billion connected devices expected by 2025, generating a global market exceeding \$6 trillion. This expansion is accompanied by an alarming surge in cyberattacks targeting IoT firmwares, with 89\% of devices presenting critical vulnerabilities. Firmware compromise attacks allow cybercriminals to gain persistent, low-level control, rendering conventional detection methods ineffective.

\section*{Problem Statement}

Current IoT firmware security solutions present several major limitations: excessive computational overhead incompatible with constrained devices, absence of continuous integrity verification, integration complexity into existing systems, and high deployment costs. There is a critical need to develop robust, performant security solutions adapted to the specific constraints of industrial IoT environments.

\section*{Objective}

This research aims to design, develop and validate SecureIoT-VIF, a modular integrity verification framework for IoT firmware security. The approach proposes a two-tier scalable architecture implemented on ESP32 platform: the \textbf{Standard Configuration}, implemented and experimentally validated, integrating software cryptography mechanisms, boot-time integrity verification, and configurable anomaly detection; and the \textbf{Expert Configuration}, designed in detail to integrate advanced mechanisms (hardware HSM, real-time verification, adaptive machine learning).

\section*{Contributions}

\textbf{Theoretical contribution:} Development of a hybrid security model combining cryptographic integrity verification and behavioral anomaly detection, with a modular architecture enabling progressive evolution from Standard Configuration to Expert Configuration.

\textbf{Methodological contribution:}
\begin{itemize}
    \item Implementation methodology for firmware security mechanisms on ESP32 with mbedTLS cryptography
    \item Security and performance evaluation framework for constrained IoT systems
    \item Two-tier modular architecture enabling progressive integration of advanced functionalities
\end{itemize}

\textbf{Technical contribution:} Complete implementation of Standard Configuration and detailed design of Expert Configuration:
\begin{itemize}
    \item \textbf{Standard Configuration (implemented)}: Firmware integrity verification with SHA-256/ECDSA cryptographic signatures, configurable anomaly detection with adaptive thresholds, extensible modular architecture, validation with DHT22 sensors
    \item \textbf{Expert Configuration (designed)}: Detailed specifications for hardware HSM, continuous real-time verification, adaptive machine learning, remote attestation
\end{itemize}

\textbf{Empirical contribution:} Rigorous experimental evaluation of Standard Configuration on representative scenarios, with comparative performance analysis, practical feasibility demonstration and attack detection effectiveness validation.

\section*{Results}

The evaluation of Standard Configuration reveals promising performance: high detection rate for firmware compromise attacks, minimal computational overhead (< 8\%), controlled energy impact, and validated modular architecture. Tests with DHT22 sensors confirm the effectiveness of the implemented approach. The Expert Configuration, designed in detail, establishes a scientifically grounded roadmap for future integration of advanced mechanisms.

\section*{Impact}

This research establishes a new approach for IoT firmware security by proposing a two-tier modular architecture that effectively balances security, performance and practicality. The implementation of Standard Configuration demonstrates the approach feasibility, while the detailed design of Expert Configuration positions SecureIoT-VIF as an evolutionary solution for IoT deployments.

\textbf{Keywords:} IoT, Firmware Security, Modular Framework, ESP32, Integrity Verification, Anomaly Detection, Embedded Systems, Cryptography, Two-Tier Architecture, Standard Configuration, Expert Configuration
