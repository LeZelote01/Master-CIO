%====================================================================
% Remerciements
%====================================================================

\chapter*{Remerciements}
\addcontentsline{toc}{chapter}{Remerciements}

Je tiens à exprimer ma profonde gratitude à toutes les personnes qui ont contribué à la réalisation de ce mémoire de recherche sur SecureIoT-VIF.

\section*{Encadrement académique}

Mes remerciements vont d'abord à mon directeur de mémoire, \textbf{[Nom du directeur]}, pour sa guidance experte, ses conseils avisés, et son soutien constant tout au long de cette recherche. Sa vision stratégique et son exigence académique ont été déterminantes pour maintenir la qualité et la rigueur de ce travail.

Je remercie également mon co-directeur, \textbf{[Nom du co-directeur]}, pour son expertise technique en sécurité IoT et ses retours constructifs qui ont enrichi considérablement l'aspect technique de cette recherche.

\section*{Communauté de recherche}

Ma gratitude s'étend aux experts et chercheurs qui ont participé à l'évaluation technique de SecureIoT-VIF. Leurs retours critiques et constructifs ont été essentiels pour valider l'efficacité et la robustesse du framework développé.

Je remercie les professionnels de l'industrie IoT qui ont accepté de tester et d'évaluer SecureIoT-VIF dans des contextes réels, fournissant des données précieuses sur la praticité et l'applicabilité du framework.

\section*{Support technique et communautaire}

Mes remerciements à la \textbf{communauté ESP-IDF et mbedTLS} pour la documentation excellente et le support technique qui ont facilité le développement de ce framework professionnel.

Je salue également la \textbf{communauté open source} en général, dont la philosophie de partage des connaissances et d'innovation collaborative a inspiré l'approche de ce projet.

\section*{Validation expérimentale}

Un merci particulier aux participants des tests expérimentaux qui ont validé la robustesse et l'efficacité de SecureIoT-VIF dans des contextes variés.

\section*{Support personnel}

Je remercie chaleureusement ma famille pour leur patience et leur soutien inconditionnel pendant les longues heures de développement et de rédaction. Leur compréhension et leurs encouragements ont été un pilier fondamental de cette réussite.

Mes amis et collègues méritent également ma reconnaissance pour leurs discussions enrichissantes, leurs suggestions d'amélioration, et leur aide lors des phases de test et de validation.

\section*{Vision collective}

Enfin, je remercie tous ceux qui partagent la vision d'une sécurité IoT robuste et accessible. Cette recherche n'aurait pas de sens sans une communauté engagée dans l'amélioration de la sécurité de notre monde numérique par l'innovation et la recherche appliquée.

Ce mémoire représente non seulement l'aboutissement d'un travail personnel, mais aussi la contribution collective de tous ces acteurs vers un objectif commun : améliorer la sécurité des firmwares IoT et établir des bases solides pour les évolutions futures du domaine.

\vspace{1cm}

\begin{flushright}
\textit{[\auteur]}\\
\textit{\annee}
\end{flushright}
