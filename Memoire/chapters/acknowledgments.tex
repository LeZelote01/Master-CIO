%====================================================================
% Remerciements
%====================================================================

\chapter*{Remerciements}
\addcontentsline{toc}{chapter}{Remerciements}

Je tiens à exprimer ma profonde gratitude à toutes les personnes qui ont contribué à la réalisation de ce mémoire de recherche sur SecureIoT-VIF Community Edition.

\section*{Encadrement académique}

Mes remerciements vont d'abord à mon directeur de mémoire, \textbf{[Nom du directeur]}, pour sa guidance experte, ses conseils avisés, et son soutien constant tout au long de cette recherche. Sa vision stratégique et son exigence académique ont été déterminantes pour maintenir la qualité et la rigueur de ce travail.

Je remercie également mon co-directeur, \textbf{[Nom du co-directeur]}, pour son expertise technique en sécurité IoT et ses retours constructifs qui ont enrichi considérablement l'aspect technique de cette recherche.

\section*{Communauté académique}

Ma gratitude s'étend aux \textbf{25 étudiants} qui ont participé à l'évaluation pédagogique de SecureIoT-VIF Community Edition. Leur engagement, leurs retours honnêtes, et leur enthousiasme pour l'apprentissage de la sécurité IoT ont été essentiels pour valider l'efficacité du framework développé.

Je remercie les \textbf{enseignants des trois établissements} qui ont accepté de tester et déployer SecureIoT-VIF Community Edition dans leurs cours, fournissant des données précieuses sur la reproductibilité et l'efficacité pédagogique du framework.

\section*{Support technique et communautaire}

Mes remerciements à la \textbf{communauté ESP-IDF et mbedTLS} pour la documentation excellente et le support technique qui ont facilité le développement de ce framework éducatif.

Je salue également la \textbf{communauté open source} en général, dont la philosophie d'accessibilité et de partage des connaissances a inspiré l'approche Community Edition de ce projet.

\section*{Validation expérimentale}

Un merci particulier aux \textbf{47 étudiants} qui ont participé aux tests de déploiement multi-sites, validant la reproductibilité de SecureIoT-VIF Community Edition dans des contextes éducatifs variés.

\section*{Support personnel}

Je remercie chaleureusement ma famille pour leur patience et leur soutien inconditionnel pendant les longues heures de développement et de rédaction. Leur compréhension et leurs encouragements ont été un pilier fondamental de cette réussite.

Mes amis et collègues méritent également ma reconnaissance pour leurs discussions enrichissantes, leurs suggestions d'amélioration, et leur aide lors des phases de test et de validation.

\section*{Financement et ressources}

Cette recherche a été réalisée avec des ressources minimales (budget total de développement : ~50€), démontrant ainsi qu'il est possible de mener des recherches impactantes même avec des contraintes budgétaires importantes. Cette contrainte s'est finalement révélée être un catalyseur d'innovation, nous poussant à développer des solutions créatives et accessibles.

\section*{Vision collective}

Enfin, je remercie tous ceux qui partagent la vision d'une éducation en cybersécurité accessible et démocratique. Cette recherche n'aurait pas de sens sans une communauté engagée dans l'amélioration de la sécurité de notre monde numérique par l'éducation et la formation.

Ce mémoire représente non seulement l'aboutissement d'un travail personnel, mais aussi la contribution collective de tous ces acteurs vers un objectif commun : rendre l'apprentissage de la sécurité IoT accessible à tous, partout dans le monde.

\vspace{1cm}

\begin{flushright}
\textit{[\auteur]}\\
\textit{\annee}
\end{flushright}