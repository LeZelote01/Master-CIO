%====================================================================
% Liste des acronymes
%====================================================================

\chapter*{Liste des acronymes}
\addcontentsline{toc}{chapter}{Liste des acronymes}

\begin{acronym}[Community]

\acro{ADM}{Anomaly Detection Module}
\acro{API}{Application Programming Interface}
\acro{AES}{Advanced Encryption Standard}
\acro{ARM}{Advanced RISC Machines}
\acro{CFI}{Control Flow Integrity}
\acro{CoAP}{Constrained Application Protocol}
\acro{CPU}{Central Processing Unit}
\acro{CVE}{Common Vulnerabilities and Exposures}
\acro{DDoS}{Distributed Denial of Service}
\acro{DHT}{Digital Humidity and Temperature}
\acro{ECDSA}{Elliptic Curve Digital Signature Algorithm}
\acro{ESP-IDF}{Espressif IoT Development Framework}
\acro{FPR}{False Positive Rate}
\acro{GPIO}{General Purpose Input/Output}
\acro{HKDF}{HMAC-based Key Derivation Function}
\acro{HSM}{Hardware Security Module}
\acro{HTTP}{HyperText Transfer Protocol}
\acro{IoT}{Internet of Things}
\acro{IVM}{Integrity Verification Module}
\acro{JSON}{JavaScript Object Notation}
\acro{KMM}{Key Management Module}
\acro{LoRaWAN}{Long Range Wide Area Network}
\acro{mbedTLS}{mbed Transport Layer Security}
\acro{ML}{Machine Learning}
\acro{MQTT}{Message Queuing Telemetry Transport}
\acro{MTTD}{Mean Time To Detection}
\acro{OTA}{Over-The-Air}
\acro{PUF}{Physically Unclonable Function}
\acro{RAM}{Remote Attestation Module}
\acro{RISC-V}{Reduced Instruction Set Computer - Five}
\acro{ROP}{Return-Oriented Programming}
\acro{RTOS}{Real-Time Operating System}
\acro{SDK}{Software Development Kit}
\acro{SE}{Secure Element}
\acro{SHA}{Secure Hash Algorithm}
\acro{SRAM}{Static Random Access Memory}
\acro{TLS}{Transport Layer Security}
\acro{TPM}{Trusted Platform Module}
\acro{TPR}{True Positive Rate}
\acro{TRNG}{True Random Number Generator}
\acro{UART}{Universal Asynchronous Receiver-Transmitter}
\acro{USB}{Universal Serial Bus}
\acro{VIF}{Verification Integrity Framework}
\acro{Wi-Fi}{Wireless Fidelity}

\end{acronym}

\vspace{1cm}

\section*{Termes spécifiques}

\textbf{Community Edition :} Version éducative et accessible de SecureIoT-VIF, conçue pour l'apprentissage avec des composants abordables et de la cryptographie software.

\textbf{Enterprise Edition :} Version avancée (théorique) utilisant des accélérateurs matériels et des fonctionnalités sophistiquées pour les déploiements industriels.

\textbf{Cryptographie Software :} Implémentation des algorithmes cryptographiques en logiciel (mbedTLS) plutôt qu'en matériel dédié.

\textbf{Seuils fixes :} Méthode de détection d'anomalies basée sur des valeurs limites prédéfinies, par opposition aux approches adaptatives ou d'apprentissage automatique.

\textbf{Overhead éducatif :} Impact acceptable sur les performances en échange d'une meilleure compréhension et observabilité des mécanismes de sécurité.

\textbf{Framework éducatif :} Outil logiciel conçu spécifiquement pour l'enseignement et l'apprentissage, privilégiant la compréhension sur l'optimisation maximale.