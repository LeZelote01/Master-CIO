%====================================================================
% Liste des acronymes
%====================================================================

\chapter*{Acronymes et Terminologie}
\addcontentsline{toc}{chapter}{Acronymes et Terminologie}

\begin{acronym}[Configuration]
\acro{ADM}{Anomaly Detection Module - Module de Détection d'Anomalies}
\acro{AES}{Advanced Encryption Standard}
\acro{API}{Application Programming Interface}
\acro{ARM}{Advanced RISC Machines}
\acro{CFI}{Control Flow Integrity}
\acro{CoAP}{Constrained Application Protocol}
\acro{CPU}{Central Processing Unit}
\acro{DHT22}{Digital Humidity and Temperature sensor}
\acro{DDoS}{Distributed Denial of Service}
\acro{DRAM}{Dynamic Random Access Memory}
\acro{ECDSA}{Elliptic Curve Digital Signature Algorithm}
\acro{ESP32}{Microcontrôleur d'Espressif Systems}
\acro{FreeRTOS}{Free Real-Time Operating System}
\acro{FPGA}{Field-Programmable Gate Array}
\acro{GPIO}{General Purpose Input/Output}
\acro{HKDF}{HMAC-based Key Derivation Function}
\acro{HMAC}{Hash-based Message Authentication Code}
\acro{HSM}{Hardware Security Module - Module de Sécurité Matériel}
\acro{HTTP}{Hypertext Transfer Protocol}
\acro{IoT}{Internet of Things - Internet des Objets}
\acro{IVM}{Integrity Verification Module - Module de Vérification d'Intégrité}
\acro{JTAG}{Joint Test Action Group}
\acro{KMM}{Key Management Module - Module de Gestion des Clés}
\acro{LoRaWAN}{Long Range Wide Area Network}
\acro{mbedTLS}{Bibliothèque cryptographique pour systèmes embarqués}
\acro{ML}{Machine Learning - Apprentissage Automatique}
\acro{MQTT}{Message Queuing Telemetry Transport}
\acro{NB-IoT}{Narrowband Internet of Things}
\acro{NVS}{Non-Volatile Storage}
\acro{OTA}{Over-The-Air (mise à jour)}
\acro{PUF}{Physically Unclonable Function}
\acro{RAM}{Random Access Memory}
\acro{RCE}{Remote Code Execution}
\acro{RISC-V}{Reduced Instruction Set Computer - Version V}
\acro{ROP}{Return-Oriented Programming}
\acro{RSA}{Rivest-Shamir-Adleman (algorithme)}
\acro{SE}{Secure Element - Élément Sécurisé}
\acro{SHA}{Secure Hash Algorithm}
\acro{SLA}{Service Level Agreement}
\acro{SSH}{Secure Shell}
\acro{TEE}{Trusted Execution Environment}
\acro{TPM}{Trusted Platform Module}
\acro{TRNG}{True Random Number Generator}
\acro{VIF}{Verification and Integrity Framework}

\end{acronym}

\section*{Terminologie spécifique}

\textbf{Configuration Standard :} Configuration implémentée et validée expérimentalement de SecureIoT-VIF, intégrant des mécanismes de sécurité basés sur la cryptographie software (mbedTLS), la vérification d'intégrité au démarrage et périodique, et la détection d'anomalies par seuils adaptatifs. Cette configuration démontre la faisabilité de l'approche sur plateforme ESP32.

\textbf{Configuration Expert :} Configuration conçue en détail pour SecureIoT-VIF, spécifiant l'intégration de mécanismes de sécurité avancés (HSM matériel, vérification temps réel continue, machine learning adaptatif, attestation distante). Cette configuration établit une roadmap scientifiquement fondée pour l'évolution future du framework.

\textbf{Overhead acceptable :} Impact mesuré sur les performances (temps d'exécution, consommation mémoire, énergie) qui reste dans des limites compatibles avec les contraintes opérationnelles des dispositifs IoT.

\textbf{Architecture modulaire :} Structure logicielle composée de modules indépendants et interchangeables, facilitant l'extension et l'adaptation du système selon les exigences spécifiques.
