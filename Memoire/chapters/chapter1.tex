%====================================================================
% Chapitre 1 : Introduction - SecureIoT-VIF Community Edition
%====================================================================

\chapter{Introduction}
\label{chap:introduction}

\section{Contexte général}

L'Internet des Objets (\ac{IoT}) représente aujourd'hui l'une des révolutions technologiques les plus significatives de notre époque. Selon les projections de l'industrie, le nombre de dispositifs \ac{IoT} connectés devrait atteindre 75 milliards d'unités d'ici 2025, générant un marché mondial estimé à plus de 6 000 milliards de dollars \cite{Statista2024IoTMarket}. Cette croissance exponentielle s'explique par l'intégration croissante de l'intelligence artificielle, l'amélioration des réseaux de communication (5G, LoRaWAN, NB-IoT), et la miniaturisation des composants électroniques.

Les dispositifs \ac{IoT} grand public englobent une vaste gamme d'appareils : objets connectés domestiques (thermostats, caméras de sécurité, assistants vocaux), dispositifs portables (montres intelligentes, trackers de fitness), appareils électroménagers intelligents, et systèmes de domotique. Ces dispositifs partagent plusieurs caractéristiques communes : ressources limitées (processeur, mémoire, stockage), contraintes énergétiques strictes, et connexion permanente aux réseaux de communication.

Cependant, cette prolifération s'accompagne d'une augmentation alarmante des cyberattaques ciblant spécifiquement les firmwares des dispositifs \ac{IoT}. Les recherches récentes de Nino et al. \cite{Nino2024UnveilingIoT} révèlent que 89\% des firmwares \ac{IoT} analysés présentent des vulnérabilités critiques, dont 67\% sont liées à l'absence de mécanismes de protection de l'intégrité du firmware. Les attaques par compromission de firmware permettent aux cybercriminels d'obtenir un contrôle persistant et de bas niveau des dispositifs, rendant les détections conventionnelles inefficaces.

\section{Défis éducatifs et de recherche en sécurité IoT}

\subsection{Barrières à l'apprentissage de la sécurité IoT}

L'enseignement et la recherche en sécurité IoT font face à plusieurs défis majeurs qui limitent l'accessibilité et l'efficacité pédagogique :

\textbf{Coût des équipements spécialisés :} Les solutions commerciales de sécurité IoT utilisent souvent des composants coûteux comme des modules de sécurité matériels (HSM) externes, des puces de chiffrement dédiées, ou des équipements de laboratoire spécialisés. Ces coûts, pouvant atteindre plusieurs centaines d'euros par dispositif, représentent une barrière significative pour les établissements d'enseignement et les projets de recherche avec des budgets limités.

\textbf{Complexité technique excessive :} Les frameworks de sécurité existants sont généralement conçus pour des applications industrielles critiques, avec des architectures complexes difficiles à comprendre pour les étudiants. Cette complexité masque souvent les concepts fondamentaux, rendant l'apprentissage progressif difficile.

\textbf{Manque d'outils pédagogiques adaptés :} Il existe peu de ressources éducatives pratiques permettant aux étudiants d'expérimenter concrètement avec les mécanismes de sécurité IoT. La plupart des formations restent théoriques, sans possibilité d'implémentation pratique sur du matériel réel.

\textbf{Absence de progressivité dans l'apprentissage :} Les solutions existantes ne permettent pas un apprentissage progressif, passant directement des concepts théoriques aux implémentations complexes, sans étapes intermédiaires permettant la consolidation des connaissances.

\subsection{Besoin d'une approche éducative accessible}

Face à ces défis, la communauté académique exprime un besoin croissant pour des solutions éducatives qui combinent :
- **Accessibilité financière** : Coût compatible avec les budgets éducatifs
- **Simplicité pédagogique** : Architecture compréhensible et modulaire
- **Praticité** : Possibilité d'expérimentation concrète sur matériel réel
- **Progressivité** : Apprentissage par étapes des concepts de sécurité

\section{Problématique de recherche}

\subsection{Vulnérabilités des firmwares IoT dans le contexte éducatif}

Les firmwares des dispositifs \ac{IoT} éducatifs présentent des vulnérabilités spécifiques qui en font d'excellents cas d'étude pour l'apprentissage. Plusieurs études récentes ont mis en évidence l'ampleur de ces vulnérabilités :

\textbf{Composants tiers vulnérables :} L'analyse de Feng et al. \cite{Feng2022OneBadApple} sur 34 136 images de firmware révèle la présence de 584 composants tiers (\ac{TPC}) associés à 128 757 vulnérabilités liées à 429 \ac{CVE}. Cette étude souligne la persistance de vulnérabilités bien connues dans les écosystèmes de firmware, créant des opportunités d'exploitation pour les attaques éducatives simulées.

\textbf{Sophistication croissante des malwares :} Les travaux de recherche récents montrent une évolution significative dans la sophistication des malwares IoT. L'étude de González-Manzano et al. \cite{Gonzalez2024ExploringShifting} révèle une augmentation de 45\% de la complexité des malwares IoT entre 2021 et 2023, nécessitant des outils éducatifs adaptés pour comprendre ces évolutions.

\textbf{Attaques par manipulation du flot de contrôle :} Les attaques de type \ac{ROP} (Return-Oriented Programming) représentent une menace particulièrement instructive pour l'apprentissage. Christou et al. \cite{Christou2024DAEDALUS} démontrent comment ces attaques peuvent être déployées, offrant d'excellents cas d'étude éducatifs.

\subsection{Limitations des solutions existantes pour l'éducation}

Les approches actuelles de sécurisation des firmwares IoT présentent plusieurs limitations pour l'usage éducatif :

\textbf{Overhead computationnel prohibitif :} Les solutions de sécurité traditionnelles, conçues pour des systèmes aux ressources abondantes, introduisent un overhead computationnel et énergétique incompatible avec les contraintes des dispositifs IoT éducatifs. Les mécanismes de vérification cryptographique conventionnels peuvent consommer jusqu'à 30\% des ressources disponibles sur un microcontrôleur de classe ARM Cortex-M \cite{Khan2024EfficiencySecurity}.

\textbf{Absence de vérification temps réel accessible :} La plupart des solutions existantes effectuent la vérification d'intégrité uniquement au démarrage du dispositif, sans permettre aux étudiants de comprendre les mécanismes de vérification continue. Cette limitation pédagogique empêche la compréhension complète des concepts de sécurité runtime.

\textbf{Sous-utilisation des capacités de base :} Les outils éducatifs existants ne permettent pas d'explorer progressivement les capacités de base des microcontrôleurs comme l'ESP32, en commençant par la cryptographie software (mbedTLS) avant d'aborder des concepts plus avancés. Cette approche progressive est pourtant essentielle pour un apprentissage efficace.

\section{Objectifs de recherche}

\subsection{Objectif principal}

L'objectif principal de cette recherche est de concevoir, développer et valider SecureIoT-VIF Community Edition, un framework éducatif de vérification d'intégrité spécialement conçu pour l'apprentissage et la recherche en sécurité des firmwares IoT. Ce framework vise à démocratiser l'accès à l'éducation en sécurité IoT en proposant une solution accessible, pratique, et pédagogiquement efficace basée sur des composants abordables (ESP32 ~8$) et des mécanismes de sécurité de base compréhensibles.

\subsection{Objectifs spécifiques}

\textbf{Objectif 1 : Analyse des menaces éducatives}
\begin{itemize}
    \item Réaliser une taxonomie des attaques par compromission de firmware adaptée au contexte éducatif
    \item Identifier les vecteurs d'attaque les plus instructifs pour l'apprentissage
    \item Évaluer les besoins pédagogiques spécifiques à l'enseignement de la sécurité IoT
\end{itemize}

\textbf{Objectif 2 : Conception d'une architecture éducative accessible}
\begin{itemize}
    \item Développer une architecture de sécurité modulaire utilisant des composants abordables (ESP32)
    \item Concevoir des mécanismes de vérification d'intégrité de base utilisant la cryptographie software (mbedTLS)
    \item Intégrer des protocoles de détection d'anomalies par seuils fixes, adaptés à l'apprentissage progressif
\end{itemize}

\textbf{Objectif 3 : Implémentation éducative et validation pédagogique}
\begin{itemize}
    \item Implémenter SecureIoT-VIF Community Edition sur plateforme ESP32 accessible (~8$)
    \item Développer des mécanismes de détection d'anomalies basés sur des seuils fixes compréhensibles
    \item Créer du matériel pédagogique et des exercices pratiques progressifs
    \item Réaliser des études comparatives avec des approches plus avancées
\end{itemize}

\textbf{Objectif 4 : Évaluation expérimentale éducative}
\begin{itemize}
    \item Évaluer l'efficacité pédagogique de détection des attaques de base sur plateforme ESP32
    \item Mesurer l'impact sur les performances et la consommation pour des contraintes éducatives
    \item Valider l'efficacité d'apprentissage auprès d'étudiants et chercheurs
    \item Établir une base méthodologique pour l'extension vers des usages plus avancés
\end{itemize}

\section{Contributions de la recherche}

Ce travail apporte plusieurs contributions significatives au domaine de l'éducation en sécurité des firmwares IoT :

\textbf{Contribution théorique :} Proposition d'un modèle de sécurité éducatif combinant vérification d'intégrité de base et détection d'anomalies par seuils fixes, spécifiquement conçu pour l'apprentissage progressif des concepts de sécurité IoT.

\textbf{Contribution méthodologique :} 
\begin{itemize}
    \item Développement d'une méthodologie d'enseignement de la sécurité IoT basée sur l'expérimentation pratique avec des composants accessibles
    \item Création d'une approche pédagogique progressive permettant la compréhension des concepts de base avant l'exploration d'implémentations avancées
    \item Proposition d'une méthodologie d'évaluation de l'efficacité éducative pour les frameworks de sécurité IoT
\end{itemize}

\textbf{Contribution technique :} Implémentation de SecureIoT-VIF Community Edition, un framework éducatif pratique offrant :
\begin{itemize}
    \item Vérification d'intégrité au démarrage avec un overhead acceptable pour l'apprentissage (< 8\%)
    \item Détection de 85\% des attaques de base dans les scénarios éducatifs
    \item Coût total de 8$ permettant l'accessibilité aux établissements d'enseignement
    \item Architecture modulaire facilitant la compréhension progressive des concepts
\end{itemize}

\textbf{Contribution empirique :} Évaluation expérimentale sur des scénarios éducatifs représentatifs, validation de l'efficacité pédagogique auprès d'étudiants, et démonstration de la faisabilité pratique de l'approche proposée pour l'enseignement de la sécurité IoT.

\section{Approche méthodologique}

\subsection{Justification de l'approche éducative}

Cette recherche adopte délibérément une approche éducative centrée sur l'accessibilité et la compréhension progressive plutôt que sur la performance maximale. Cette stratégie méthodologique se justifie par :

\textbf{Priorité à l'apprentissage :} Une implémentation claire et compréhensible des mécanismes de sécurité de base offre une valeur pédagogique supérieure à une optimisation complexe difficile à comprendre.

\textbf{Reproductibilité éducative :} La focalisation sur ESP32, plateforme largement accessible et documentée, facilite la reproduction des expériences dans différents établissements d'enseignement.

\textbf{Optimisation des ressources éducatives :} L'allocation des ressources à une implémentation compréhensible et bien documentée permet d'atteindre des niveaux d'efficacité pédagogique qui ne seraient pas atteignables avec une approche purement technique.

\textbf{Base pour progression :} L'approche établit une base méthodologique solide pour la progression vers des implémentations plus avancées une fois les concepts de base maîtrisés.

\subsection{Méthodologie de validation}

\textbf{Phase 1 - Analyse et état de l'art :} Revue systématique de la littérature, analyse des besoins éducatifs, et identification des lacunes dans les outils d'enseignement existants.

\textbf{Phase 2 - Conception et modélisation éducative :} Développement du modèle de sécurité éducatif, conception de l'architecture du framework accessible, et spécification des protocoles de sécurité adaptés à l'apprentissage.

\textbf{Phase 3 - Implémentation éducative :} Implémentation complète du framework sur ESP32, développement d'outils d'évaluation pédagogiques, et création de matériel éducatif associé.

\textbf{Phase 4 - Évaluation expérimentale et pédagogique :} Tests de sécurité sur scénarios éducatifs, mesures de performance compatibles avec les contraintes d'apprentissage, et validation de l'efficacité pédagogique.

\section{Organisation du mémoire}

Ce mémoire est organisé en sept chapitres :

\textbf{Chapitre 1 - Introduction :} Présente le contexte éducatif, la problématique, les objectifs et les contributions de la recherche, avec justification de l'approche Community Edition.

\textbf{Chapitre 2 - État de l'art :} Analyse les travaux existants en sécurité des firmwares IoT, les mécanismes de vérification d'intégrité, et les approches éducatives dans le domaine.

\textbf{Chapitre 3 - Analyse des menaces :} Développe une taxonomie des attaques par compromission de firmware adaptée au contexte éducatif et présente un modèle de menaces pour l'apprentissage.

\textbf{Chapitre 4 - Conception du framework :} Détaille l'architecture de SecureIoT-VIF Community Edition, les mécanismes de sécurité de base proposés, et les choix de conception orientés apprentissage.

\textbf{Chapitre 5 - Implémentation :} Présente l'implémentation du framework sur ESP32, les optimisations réalisées pour l'usage éducatif, et le matériel pédagogique développé.

\textbf{Chapitre 6 - Évaluation et résultats :} Analyse les résultats de l'évaluation expérimentale et présente la validation de l'efficacité pédagogique.

\textbf{Chapitre 7 - Conclusion et perspectives :} Synthétise les contributions, discute les limitations de l'approche éducative, et propose des directions pour l'évolution vers des implémentations plus avancées.

\section{Délimitations de l'étude}

\subsection{Périmètre de l'étude éducative}

Cette recherche se concentre délibérément sur :

\textbf{Plateforme cible accessible :} ESP32 comme plateforme principale d'évaluation, choisie pour son excellent rapport coût/performance (~8$) et sa large adoption dans l'écosystème éducatif.

\textbf{Scope fonctionnel éducatif :} Mécanismes de sécurité de base (crypto software, vérification au démarrage, détection par seuils) privilégiant la compréhension sur la performance.

\textbf{Scénarios d'attaque éducatifs :} Attaques représentatives et instructives, sélectionnées pour leur valeur pédagogique plutôt que pour leur sophistication technique.

\subsection{Extensions futures envisagées}

L'approche éducative établit les fondations pour des extensions futures :

\textbf{Progression vers l'advanced :} Extension vers des mécanismes plus sophistiqués une fois les concepts de base maîtrisés (HSM matériels, vérification temps réel, ML adaptatif).

\textbf{Diversité des plateformes :} Adaptation vers d'autres plateformes éducatives (Arduino, Raspberry Pi) pour élargir l'accessibilité.

\textbf{Déploiements éducatifs à plus grande échelle :} Extension vers des environnements de laboratoire multi-dispositifs pour l'apprentissage de concepts distribués.

Cette approche méthodologique assure une contribution significative à l'éducation en sécurité IoT tout en établissant une roadmap claire pour la progression vers des applications plus avancées, permettant aux étudiants et chercheurs de développer une compréhension solide avant d'aborder des concepts plus complexes.