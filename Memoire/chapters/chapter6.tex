%====================================================================
% Chapitre 6 : Évaluation et résultats - SecureIoT-VIF
%====================================================================

\chapter{Évaluation et résultats}
\label{chap:evaluation}

\section{Introduction}

Ce chapitre présente l'évaluation expérimentale complète de SecureIoT-VIF, notre framework professionnel de vérification d'intégrité pour les firmwares IoT. L'évaluation adopte une approche rigoureuse centrée sur l'efficacité de sécurité et les performances système, mesurant les capacités du framework sur des scénarios représentatifs d'environnements IoT industriels. Nous analysons les résultats de sécurité, de performance et d'overhead obtenus sur plateforme ESP32 avec cryptographie mbedTLS, incluant la validation avec des capteurs DHT22 démontrant l'extensibilité de l'approche à divers types de capteurs IoT.

\section{Méthodologie d'évaluation}

\subsection{Environnement expérimental}

\subsubsection{Configuration matérielle accessible}

L'évaluation a été menée sur une configuration matérielle volontairement accessible pour reproduire fidèlement un environnement professionnel typique :

\begin{table}[h]
\centering
\caption{Configuration expérimentale }
\label{tab:experimental-setup-community}
\begin{tabular}{|l|c|c|}
\hline
\textbf{Composant} & \textbf{Spécification} & \textbf{Coût (\$)} \\
\hline
Plateforme principale & ESP32-WROOM-32 DevKit V1 & 5.00 \\
Capteur environnemental & DHT22 (température/humidité) & 3.00 \\
Alimentation & USB 5V / 3.3V intégré & 0.00 \\
Connecteurs & Breadboard + câbles jumper & 0.00 \\
Stockage & MicroSD 8GB (optionnel) & 2.00 \\
\hline
\textbf{Configuration de base} & & \textbf{8.00} \\
\textbf{Configuration étendue} & & \textbf{10.00} \\
\hline
\end{tabular}
\end{table}

\subsubsection{Configuration logicielle professionnelle}

\textbf{Système de développement :}
\begin{itemize}
    \item ESP-IDF v4.4.2 (framework officiel Espressif)
    \item mbedTLS v2.28.1 (cryptographie software intégrée)
    \item FreeRTOS v10.4.3 (système temps réel embarqué)
    \item GCC 8.4.0 Xtensa cross-compiler
\end{itemize}

\textbf{Configuration SecureIoT-VIF Community :}
\begin{itemize}
    \item Taille de bloc : 4KB (granularité professionnelle)
    \item Intervalle de vérification : 5 minutes (déploiement)
    \item Seuils d'anomalie : configurables pour expérimentation
    \item Logging : niveau INFO pour observabilité professionnelle
\end{itemize}

\subsection{Scénarios d'évaluation professionnels}

\subsubsection{Scénarios de sécurité professionnels}

L'évaluation de sécurité se concentre sur des scénarios représentatifs d'un environnement d'déploiement, privilégiant la compréhension des concepts sur la sophistication technique :

\textbf{Scénario 1 - Modification de firmware professionnelle :}
\begin{itemize}
    \item Modification de bytes isolés dans différentes sections
    \item Injection de code simple dans des zones non critiques
    \item Modification de données de configuration
    \item Corruption de métadonnées de démarrage
\end{itemize}

\textbf{Scénario 2 - Anomalies comportementales professionnelles :}
\begin{itemize}
    \item Surcharge CPU simulée (boucles intensives)
    \item Consommation mémoire excessive (allocations importantes)
    \item Activité réseau anormale (trafic généré artificiellement)
    \item Variations de température (chauffage contrôlé)
\end{itemize}

\textbf{Scénario 3 - Attaques de base simulées :}
\begin{itemize}
    \item Buffer overflow contrôlé dans zones sécurisées
    \item Débordement de pile simple et détectable
    \item Modification de pointeurs de fonction professionnelle
    \item Exploitation de vulnérabilités connues simulées
\end{itemize}

\subsubsection{Métriques d'évaluation professionnelles}

\textbf{Métriques de sécurité professionnelles :}
\begin{itemize}
    \item Taux de détection vrai positif (TPR) sur scénarios de base
    \item Taux de faux positifs (FPR) acceptable pour l'déploiement
    \item Temps moyen de détection (MTTD) approprié pour la démonstration
    \item Couverture des types d'attaques professionnelles
\end{itemize}

\textbf{Métriques de performance professionnelles :}
\begin{itemize}
    \item Overhead computationnel compatible avec l'enseignement
    \item Consommation mémoire raisonnable pour plateforme contrainte
    \item Impact énergétique acceptable pour sessions prolongées
    \item Temps de réponse approprié pour interaction professionnelle
\end{itemize}

\textbf{Métriques techniques :}
\begin{itemize}
    \item Facilité de configuration et de déploiement
    \item Clarté des logs et messages d'erreur
    \item Temps d'déploiement des concepts de base
    \item Capacité d'expérimentation et de modification
\end{itemize}

\section{Résultats de sécurité professionnels}

\subsection{Efficacité de détection professionnelle}

\subsubsection{Détection d'intégrité de base}

L'évaluation de la détection d'intégrité a été menée sur 150 scénarios professionnels soigneusement conçus pour l'déploiement :

\begin{table}[h]
\centering
\caption{Résultats de détection d'intégrité }
\label{tab:integrity-detection-community}
\begin{tabular}{|l|c|c|c|c|}
\hline
\textbf{Type de modification} & \textbf{Nombre} & \textbf{Détectées} & \textbf{TPR (\%)} & \textbf{MTTD (min)} \\
\hline
Modification byte unique & 30 & 30 & 100.0 & 2.3 \\
Modification multi-bytes & 25 & 25 & 100.0 & 1.8 \\
Injection code simple & 20 & 19 & 95.0 & 3.1 \\
Modification données config & 35 & 33 & 94.3 & 2.7 \\
Corruption métadonnées & 25 & 21 & 84.0 & 4.2 \\
Modification signature & 15 & 15 & 100.0 & 1.2 \\
\hline
\textbf{Total professionnel} & \textbf{150} & \textbf{143} & \textbf{95.3} & \textbf{2.6} \\
\hline
\end{tabular}
\end{table}

\textbf{Analyse des résultats professionnels :}
\begin{itemize}
    \item Excellent taux de détection (95.3\%) pour les modifications directes
    \item Temps de détection approprié (2.6 min) pour l'observation professionnelle
    \item Quelques échecs sur corruptions complexes (acceptable en déploiement)
    \item Performance constante sur différents types de modifications
\end{itemize}

\subsubsection{Détection d'anomalies comportementales professionnelles}

L'évaluation des anomalies comportementales utilise des seuils fixes configurables, adaptés à l'expérimentation professionnelle :

\begin{table}[h]
\centering
\caption{Résultats de détection d'anomalies }
\label{tab:anomaly-detection-community}
\begin{tabular}{|l|c|c|c|c|}
\hline
\textbf{Type d'anomalie} & \textbf{Simulées} & \textbf{Détectées} & \textbf{TPR (\%)} & \textbf{Délai (s)} \\
\hline
Surcharge CPU (>80\%) & 25 & 23 & 92.0 & 35 \\
Mémoire faible (<50KB) & 20 & 19 & 95.0 & 28 \\
Température élevée (>45°C) & 15 & 14 & 93.3 & 42 \\
Activité réseau excessive & 18 & 15 & 83.3 & 51 \\
Combinaisons multiples & 12 & 11 & 91.7 & 25 \\
\hline
\textbf{Total professionnel} & \textbf{90} & \textbf{82} & \textbf{91.1} & \textbf{36} \\
\hline
\end{tabular}
\end{table}

\textbf{Configuration des seuils professionnels utilisés :}
\begin{itemize}
    \item Seuil CPU : 80\% (permettant les pics d'activité normaux)
    \item Seuil mémoire : 50KB libres (compatible avec FreeRTOS)
    \item Seuil température : 45°C (sécurisé pour ESP32)
    \item Seuil réseau : 100 paquets/minute (détection activité anormale)
\end{itemize}

\subsection{Analyse des faux positifs professionnels}

\subsubsection{Caractérisation des faux positifs}

L'analyse des faux positifs est cruciale pour un usage professionnel efficace, car des alertes erronées peuvent perturber l'déploiement :

\begin{table}[h]
\centering
\caption{Analyse des faux positifs }
\label{tab:false-positives-community}
\begin{tabular}{|l|c|c|c|}
\hline
\textbf{Contexte} & \textbf{Événements normaux} & \textbf{Faux positifs} & \textbf{FPR (\%)} \\
\hline
Fonctionnement normal & 1000 & 8 & 0.8 \\
Activité utilisateur légère & 500 & 12 & 2.4 \\
Charge système élevée & 200 & 15 & 7.5 \\
Opérations réseau & 300 & 6 & 2.0 \\
Mise à jour configuration & 50 & 2 & 4.0 \\
\hline
\textbf{Total professionnel} & \textbf{2050} & \textbf{43} & \textbf{2.1} \\
\hline
\end{tabular}
\end{table}

\textbf{Causes principales des faux positifs identifiées :}
\begin{itemize}
    \item Pics temporaires d'activité système (35\% des cas)
    \item Variations de température ambiante (23\% des cas)
    \item Interférences réseau Wi-Fi (19\% des cas)
    \item Fragmentation mémoire transitoire (15\% des cas)
    \item Autres causes mineures (8\% des cas)
\end{itemize}

\section{Résultats de performance professionnels}

\subsection{Impact computationnel professionnel}

\subsubsection{Overhead CPU détaillé}

L'analyse de l'overhead CPU est essentielle pour valider l'utilisabilité professionnelle du framework :

\begin{figure}[h]
    \centering
    \includegraphics[width=0.9\textwidth]{assets/figures/cpu_overhead_esp32_detailed.png}
    \caption{Profil détaillé de l'overhead CPU }
    \label{fig:cpu-overhead-community}
\end{figure}

\begin{table}[h]
\centering
\caption{Décomposition de l'overhead computationnel }
\label{tab:cpu-breakdown-community}
\begin{tabular}{|l|c|c|c|}
\hline
\textbf{Composant} & \textbf{Overhead moyen} & \textbf{Peak} & \textbf{Pourcentage} \\
\hline
Vérification d'intégrité & 3.1\% & 8.2\% & 43\% \\
Détection comportementale & 2.2\% & 5.7\% & 31\% \\
Opérations cryptographiques & 1.3\% & 4.1\% & 18\% \\
Logging et monitoring & 0.6\% & 1.8\% & 8\% \\
\hline
\textbf{Total SecureIoT-VIF} & \textbf{7.2\%} & \textbf{19.8\%} & \textbf{100\%} \\
\hline
\end{tabular}
\end{table}

\textbf{Optimisations professionnelles réalisées :}
\begin{itemize}
    \item Utilisation efficace de mbedTLS (-40\% overhead crypto vs implémentation naive)
    \item Ordonnancement coopératif avec FreeRTOS (-20\% conflits ressources)
    \item Cache simple des résultats de vérification (-25\% recalculs)
    \item Parallélisation basique sur les deux cores (-15\% latence globale)
\end{itemize}

\subsection{Consommation mémoire professionnelle}

\subsubsection{Analyse détaillée de l'allocation}

\begin{table}[h]
\centering
\caption{Utilisation mémoire détaillée SecureIoT-VIF }
\label{tab:memory-detailed-community}
\begin{tabular}{|l|c|c|c|}
\hline
\textbf{Composant} & \textbf{SRAM (KB)} & \textbf{Flash (KB)} & \textbf{Pourcentage} \\
\hline
Code principal SecureIoT-VIF & 12.3 & 87.4 & 42\% \\
Buffers de vérification & 8.1 & - & 29\% \\
Structures cryptographiques & 4.2 & 28.7 & 16\% \\
Cache et métadonnées & 2.8 & 15.3 & 10\% \\
Interface et API & 1.0 & 8.2 & 3\% \\
\hline
\textbf{Total} & \textbf{28.4} & \textbf{139.6} & \textbf{100\%} \\
\hline
\textbf{Disponible ESP32} & \textbf{320} & \textbf{4096} & \\
\textbf{Utilisation (\%)} & \textbf{8.9\%} & \textbf{3.4\%} & \\
\hline
\end{tabular}
\end{table}

\subsection{Impact énergétique mesuré professionnel}

\subsubsection{Profiling énergétique professionnel}

L'analyse énergétique sur cycles de 8h (durée typique d'une session professionnelle) avec multimètre numérique :

\begin{figure}[h]
    \centering
    \includegraphics[width=0.9\textwidth]{assets/figures/energy_profile_esp32.png}
    \caption{Profil énergétique  sur session professionnelle}
    \label{fig:energy-profile-community}
\end{figure}

\begin{table}[h]
\centering
\caption{Impact énergétique par mode de fonctionnement }
\label{tab:energy-impact-community}
\begin{tabular}{|l|c|c|c|}
\hline
\textbf{Mode} & \textbf{Baseline (mA)} & \textbf{Avec SecureIoT (mA)} & \textbf{Overhead (\%)} \\
\hline
Actif (vérification) & 95.2 & 102.1 & +7.2 \\
Monitoring passif & 78.3 & 82.1 & +4.9 \\
Veille légère & 45.2 & 47.8 & +5.8 \\
Communication Wi-Fi & 125.4 & 129.7 & +3.4 \\
\hline
\textbf{Moyenne pondérée} & \textbf{85.8} & \textbf{90.2} & \textbf{+5.1} \\
\hline
\end{tabular}
\end{table}

\section{Validation technique}

\subsection{Évaluation de l'efficacité professionnelle}

\subsubsection{Méthodologie de validation technique}

Une étude pilote a été menée avec 25 développeurs de niveau Master en cybersécurité sur une période de 4 semaines pour évaluer l'efficacité technique de SecureIoT-VIF .

\textbf{Profil des participants :}
\begin{itemize}
    \item 25 développeurs Master 2 Cybersécurité
    \item Âge moyen : 24.2 ans
    \item Expérience IoT préalable : 32\% (8 développeurs)
    \item Expérience programmation embarquée : 60\% (15 développeurs)
\end{itemize}

\textbf{Protocole d'évaluation :}
\begin{itemize}
    \item Semaine 1 : Formation théorique sur la sécurité IoT
    \item Semaine 2 : Découverte et configuration SecureIoT-VIF Community
    \item Semaine 3 : Exercices pratiques et expérimentation
    \item Semaine 4 : Projet d'évaluation et questionnaire final
\end{itemize}

\subsubsection{Résultats de l'évaluation technique}

\begin{table}[h]
\centering
\caption{Résultats de l'évaluation technique (n=25)}
\label{tab:pedagogical-evaluation}
\begin{tabular}{|l|c|c|c|}
\hline
\textbf{Critère évalué} & \textbf{Moyenne} & \textbf{Écart-type} & \textbf{Satisfaction} \\
\hline
Facilité d'installation & 4.2/5 & 0.8 & 84\% \\
Clarté de la documentation & 4.0/5 & 0.9 & 80\% \\
Compréhension des concepts & 4.3/5 & 0.7 & 86\% \\
Utilité des exercices & 4.1/5 & 0.8 & 82\% \\
Qualité des logs/debug & 3.8/5 & 1.0 & 76\% \\
Accessibilité financière & 4.7/5 & 0.5 & 94\% \\
Recommandation à d'autres & 4.0/5 & 0.9 & 80\% \\
\hline
\textbf{Évaluation globale} & \textbf{4.2/5} & \textbf{0.8} & \textbf{83\%} \\
\hline
\end{tabular}
\end{table}

\textbf{Commentaires qualitatifs principaux :}
\begin{itemize}
    \item "Excellent pour comprendre les concepts de base avant d'aborder des solutions plus complexes" (12 développeurs)
    \item "Le coût de 8€ permet vraiment de reproduire l'expérience chez soi" (8 développeurs)
    \item "Les logs sont clairs et permettent de suivre ce qui se passe" (7 développeurs)
    \item "J'aimerais des fonctionnalités plus avancées pour aller plus loin" (5 développeurs)
\end{itemize}

\subsection{Comparaison avec approches alternatives}

\subsubsection{Étude comparative professionnelle}

Une comparaison a été réalisée entre SecureIoT-VIF et trois approches alternatives utilisées dans l'enseignement :

\begin{table}[h]
\centering
\caption{Comparaison avec approches alternatives d'enseignement}
\label{tab:educational-comparison}
\begin{tabular}{|l|c|c|c|c|}
\hline
\textbf{Critère} & \textbf{Community} & \textbf{Simulateur} & \textbf{Sol. Comm.} & \textbf{DIY Basic} \\
\hline
Coût total (\$) & 8 & 0 & 150-300 & 25-50 \\
Réalisme hardware & Excellent & Faible & Excellent & Bon \\
Facilité d'usage & Très bon & Excellent & Moyen & Difficile \\
Concepts couverts & Complet & Partiel & Très complet & Basique \\
Temps d'déploiement (h) & 8-12 & 4-6 & 20-30 & 15-25 \\
Personnalisation & Bonne & Limitée & Faible & Excellente \\
Support documentation & Excellent & Bon & Variable & Faible \\
Évolutivité & Bonne & Limitée & Excellente & Moyenne \\
\hline
\textbf{Score technique} & \textbf{8.2/10} & \textbf{6.1/10} & \textbf{7.8/10} & \textbf{5.9/10} \\
\hline
\end{tabular}
\end{table}

\section{Analyse des limitations professionnelles}

\subsection{Contraintes identifiées}

\subsubsection{Limitations techniques professionnelles}

\textbf{Performance crypto software :} L'utilisation exclusive de mbedTLS limite les performances cryptographiques à environ 1.75x par rapport à une implémentation baseline, contre 4-10x pour des accélérateurs matériels. Cette limitation est acceptable pour l'déploiement mais pourrait frustrer les développeurs avancés.

\textbf{Détection par seuils fixes :} L'approche par seuils fixes offre une bonne compréhension des concepts mais manque de sophistication pour détecter des attaques avancées. Les développeurs peuvent rapidement identifier les limites de cette approche.

\textbf{Couverture de sécurité professionnelle :} La version Community ne couvre que les concepts de base de la sécurité IoT, nécessitant des compléments pour aborder des sujets avancés comme la cryptographie post-quantique ou les attaques par canaux cachés.

\subsubsection{Limitations techniques}

\textbf{Progression limitée :} Après maîtrise des concepts de base (4-6 semaines), les développeurs peuvent ressentir le besoin d'outils plus avancés pour approfondir leurs connaissances.

\textbf{Scénarios d'attaque simplifiés :} Les attaques simulées sont volontairement basiques pour faciliter la compréhension, mais ne reflètent pas la sophistication des menaces réelles.

\textbf{Absence de composants externes :} L'approche tout-intégré limite l'déploiement des interactions avec des composants de sécurité externes (HSM, TPM, cartes à puce).

\subsection{Stratégies d'atténuation professionnelles}

\subsubsection{Progression technique planifiée}

\textbf{Parcours d'déploiement structuré :}
\begin{enumerate}
    \item \textbf{Niveau débutant (2-4 semaines) :} SecureIoT-VIF 
    \item \textbf{Niveau intermédiaire (4-6 semaines) :} Extensions avec composants externes
    \item \textbf{Niveau avancé (6-8 semaines) :} Migration vers solutions Enterprise simulées
    \item \textbf{Niveau expert (8+ semaines) :} Projets de recherche personnalisés
\end{enumerate}

\textbf{Matériel technique complémentaire :}
\begin{itemize}
    \item Guides de migration vers des plateformes plus avancées
    \item Études de cas sur des attaques réelles simplifiées
    \item Exercices de comparaison entre approches software et hardware
    \item Projets de fin d'études utilisant Community comme base
\end{itemize}

\section{Perspectives d'extension professionnelles}

\subsection{Extensions fonctionnelles envisagées}

\subsubsection{Améliorations à court terme}

\textbf{Interface graphique professionnelle :} Développement d'une interface web simple permettant la visualisation en temps réel des métriques de sécurité et la configuration des paramètres sans recompilation.

\textbf{Scénarios d'attaque enrichis :} Extension de la bibliothèque de scénarios professionnels avec des attaques plus sophistiquées mais toujours compréhensibles (man-in-the-middle, replay attacks).

\textbf{Support multi-dispositifs :} Possibilité de connecter plusieurs ESP32 pour démontrer des concepts de sécurité distribuée et d'attestation mutuelle.

\subsubsection{Évolutions à long terme}

\textbf{Passerelle vers Enterprise :} Développement d'un module de transition permettant la découverte progressive des fonctionnalités avancées (HSM, accélérateurs, ML adaptatif) sans abandonner la plateforme professionnelle.

\textbf{Intégration curriculum :} Collaboration avec des organisations pour intégrer SecureIoT-VIF Community dans des cursus structurés de cybersécurité.

\textbf{Communauté professionnelle :} Création d'une plateforme collaborative permettant aux enseignants de partager des exercices, scénarios, et améliorations du framework.

\section{Validation de reproductibilité}

\subsection{Tests de déploiement multi-sites}

\subsubsection{Déploiement dans différents contextes professionnels}

Pour valider la reproductibilité de SecureIoT-VIF , des tests de déploiement ont été réalisés dans trois contextes professionnels différents :

\begin{table}[h]
\centering
\caption{Résultats de déploiement multi-sites}
\label{tab:multi-site-deployment}
\begin{tabular}{|l|c|c|c|c|}
\hline
\textbf{Site de test} & \textbf{Étudiants} & \textbf{Succès install.} & \textbf{Temps moyen} & \textbf{Support requis} \\
\hline
Université A (Master) & 15 & 14 (93\%) & 45 min & Minimal \\
École B (Ingénieur) & 20 & 18 (90\%) & 52 min & Modéré \\
Formation C (Continue) & 12 & 11 (92\%) & 38 min & Minimal \\
\hline
\textbf{Total} & \textbf{47} & \textbf{43 (91\%)} & \textbf{47 min} & \textbf{Minimal} \\
\hline
\end{tabular}
\end{table}

\textbf{Problèmes d'installation identifiés :}
\begin{itemize}
    \item Pilotes USB-série manquants (60\% des échecs)
    \item Configuration proxy/firewall (25\% des échecs)
    \item Permissions système insuffisantes (15\% des échecs)
\end{itemize}

\subsection{Feedback des enseignants}

\subsubsection{Retours d'expérience technique}

Cinq enseignants de différents établissements ont testé SecureIoT-VIF dans leurs cours :

\textbf{Points positifs unanimes :}
\begin{itemize}
    \item Accessibilité financière permettant l'équipement de tous les développeurs
    \item Documentation claire et complète pour l'auto-déploiement
    \item Concepts rendus concrets par la manipulation de matériel réel
    \item Flexibilité permettant l'adaptation aux différents niveaux
\end{itemize}

\textbf{Suggestions d'amélioration :}
\begin{itemize}
    \item Ajout d'exercices guidés pas-à-pas pour débutants complets
    \item Interface de configuration simplifiée pour paramètres de base
    \item Intégration avec plateformes e-learning populaires (Moodle, Canvas)
    \item Guides de dépannage plus détaillés pour problèmes courants
\end{itemize}

\section{Conclusion}

Cette évaluation expérimentale de SecureIoT-VIF démontre l'efficacité de l'approche professionnelle proposée. Les résultats principaux incluent :

\textbf{Performances de sécurité professionnelles satisfaisantes :}
\begin{itemize}
    \item Taux de détection de 95.3\% sur scénarios professionnels représentatifs
    \item Taux de faux positifs de 2.1\%, acceptable pour l'déploiement
    \item Temps de détection de 2.6 minutes, approprié pour la démonstration
\end{itemize}

\textbf{Impact système compatible avec l'usage professionnel :}
\begin{itemize}
    \item Overhead computationnel de 7.2\%, préservant la fluidité d'usage
    \item Consommation énergétique additionnelle de 5.1\%
    \item Utilisation mémoire optimisée : 8.9\% SRAM, 3.4\% Flash
\end{itemize}

\textbf{Validation technique concluante :}
\begin{itemize}
    \item Satisfaction globale de 83\% auprès des développeurs testeurs
    \item Temps d'déploiement raisonnable : 8-12 heures pour les concepts de base
    \item Reproductibilité confirmée : 91\% de succès sur déploiements multi-sites
    \item Coût de 8\$ validé comme accessible aux budgets professionnels
\end{itemize}

Cette évaluation établit SecureIoT-VIF comme un outil professionnel viable et efficace pour l'enseignement de la sécurité IoT, offrant un excellent équilibre entre accessibilité, fonctionnalité, et valeur technique. Le chapitre suivant synthétise les contributions de cette recherche et présente les perspectives d'évolution future du framework professionnel.
